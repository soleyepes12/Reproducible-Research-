% Options for packages loaded elsewhere
\PassOptionsToPackage{unicode}{hyperref}
\PassOptionsToPackage{hyphens}{url}
%
\documentclass[
]{article}
\usepackage{lmodern}
\usepackage{amssymb,amsmath}
\usepackage{ifxetex,ifluatex}
\ifnum 0\ifxetex 1\fi\ifluatex 1\fi=0 % if pdftex
  \usepackage[T1]{fontenc}
  \usepackage[utf8]{inputenc}
  \usepackage{textcomp} % provide euro and other symbols
\else % if luatex or xetex
  \usepackage{unicode-math}
  \defaultfontfeatures{Scale=MatchLowercase}
  \defaultfontfeatures[\rmfamily]{Ligatures=TeX,Scale=1}
\fi
% Use upquote if available, for straight quotes in verbatim environments
\IfFileExists{upquote.sty}{\usepackage{upquote}}{}
\IfFileExists{microtype.sty}{% use microtype if available
  \usepackage[]{microtype}
  \UseMicrotypeSet[protrusion]{basicmath} % disable protrusion for tt fonts
}{}
\makeatletter
\@ifundefined{KOMAClassName}{% if non-KOMA class
  \IfFileExists{parskip.sty}{%
    \usepackage{parskip}
  }{% else
    \setlength{\parindent}{0pt}
    \setlength{\parskip}{6pt plus 2pt minus 1pt}}
}{% if KOMA class
  \KOMAoptions{parskip=half}}
\makeatother
\usepackage{xcolor}
\IfFileExists{xurl.sty}{\usepackage{xurl}}{} % add URL line breaks if available
\IfFileExists{bookmark.sty}{\usepackage{bookmark}}{\usepackage{hyperref}}
\hypersetup{
  hidelinks,
  pdfcreator={LaTeX via pandoc}}
\urlstyle{same} % disable monospaced font for URLs
\usepackage[margin=1in]{geometry}
\usepackage{color}
\usepackage{fancyvrb}
\newcommand{\VerbBar}{|}
\newcommand{\VERB}{\Verb[commandchars=\\\{\}]}
\DefineVerbatimEnvironment{Highlighting}{Verbatim}{commandchars=\\\{\}}
% Add ',fontsize=\small' for more characters per line
\usepackage{framed}
\definecolor{shadecolor}{RGB}{248,248,248}
\newenvironment{Shaded}{\begin{snugshade}}{\end{snugshade}}
\newcommand{\AlertTok}[1]{\textcolor[rgb]{0.94,0.16,0.16}{#1}}
\newcommand{\AnnotationTok}[1]{\textcolor[rgb]{0.56,0.35,0.01}{\textbf{\textit{#1}}}}
\newcommand{\AttributeTok}[1]{\textcolor[rgb]{0.77,0.63,0.00}{#1}}
\newcommand{\BaseNTok}[1]{\textcolor[rgb]{0.00,0.00,0.81}{#1}}
\newcommand{\BuiltInTok}[1]{#1}
\newcommand{\CharTok}[1]{\textcolor[rgb]{0.31,0.60,0.02}{#1}}
\newcommand{\CommentTok}[1]{\textcolor[rgb]{0.56,0.35,0.01}{\textit{#1}}}
\newcommand{\CommentVarTok}[1]{\textcolor[rgb]{0.56,0.35,0.01}{\textbf{\textit{#1}}}}
\newcommand{\ConstantTok}[1]{\textcolor[rgb]{0.00,0.00,0.00}{#1}}
\newcommand{\ControlFlowTok}[1]{\textcolor[rgb]{0.13,0.29,0.53}{\textbf{#1}}}
\newcommand{\DataTypeTok}[1]{\textcolor[rgb]{0.13,0.29,0.53}{#1}}
\newcommand{\DecValTok}[1]{\textcolor[rgb]{0.00,0.00,0.81}{#1}}
\newcommand{\DocumentationTok}[1]{\textcolor[rgb]{0.56,0.35,0.01}{\textbf{\textit{#1}}}}
\newcommand{\ErrorTok}[1]{\textcolor[rgb]{0.64,0.00,0.00}{\textbf{#1}}}
\newcommand{\ExtensionTok}[1]{#1}
\newcommand{\FloatTok}[1]{\textcolor[rgb]{0.00,0.00,0.81}{#1}}
\newcommand{\FunctionTok}[1]{\textcolor[rgb]{0.00,0.00,0.00}{#1}}
\newcommand{\ImportTok}[1]{#1}
\newcommand{\InformationTok}[1]{\textcolor[rgb]{0.56,0.35,0.01}{\textbf{\textit{#1}}}}
\newcommand{\KeywordTok}[1]{\textcolor[rgb]{0.13,0.29,0.53}{\textbf{#1}}}
\newcommand{\NormalTok}[1]{#1}
\newcommand{\OperatorTok}[1]{\textcolor[rgb]{0.81,0.36,0.00}{\textbf{#1}}}
\newcommand{\OtherTok}[1]{\textcolor[rgb]{0.56,0.35,0.01}{#1}}
\newcommand{\PreprocessorTok}[1]{\textcolor[rgb]{0.56,0.35,0.01}{\textit{#1}}}
\newcommand{\RegionMarkerTok}[1]{#1}
\newcommand{\SpecialCharTok}[1]{\textcolor[rgb]{0.00,0.00,0.00}{#1}}
\newcommand{\SpecialStringTok}[1]{\textcolor[rgb]{0.31,0.60,0.02}{#1}}
\newcommand{\StringTok}[1]{\textcolor[rgb]{0.31,0.60,0.02}{#1}}
\newcommand{\VariableTok}[1]{\textcolor[rgb]{0.00,0.00,0.00}{#1}}
\newcommand{\VerbatimStringTok}[1]{\textcolor[rgb]{0.31,0.60,0.02}{#1}}
\newcommand{\WarningTok}[1]{\textcolor[rgb]{0.56,0.35,0.01}{\textbf{\textit{#1}}}}
\usepackage{graphicx,grffile}
\makeatletter
\def\maxwidth{\ifdim\Gin@nat@width>\linewidth\linewidth\else\Gin@nat@width\fi}
\def\maxheight{\ifdim\Gin@nat@height>\textheight\textheight\else\Gin@nat@height\fi}
\makeatother
% Scale images if necessary, so that they will not overflow the page
% margins by default, and it is still possible to overwrite the defaults
% using explicit options in \includegraphics[width, height, ...]{}
\setkeys{Gin}{width=\maxwidth,height=\maxheight,keepaspectratio}
% Set default figure placement to htbp
\makeatletter
\def\fps@figure{htbp}
\makeatother
\setlength{\emergencystretch}{3em} % prevent overfull lines
\providecommand{\tightlist}{%
  \setlength{\itemsep}{0pt}\setlength{\parskip}{0pt}}
\setcounter{secnumdepth}{-\maxdimen} % remove section numbering

\author{}
\date{\vspace{-2.5em}}

\begin{document}

\hypertarget{reproducible-research}{%
\section{Reproducible Research}\label{reproducible-research}}

This assignment makes use of data from a personal activity monitoring
device. This device collects data at 5 minute intervals through out the
day. The data consists of two months of data from an anonymous
individual collected during the months of October and November, 2012 and
include the number of steps taken in 5 minute intervals each day.

\hypertarget{loading-and-processing-the-data}{%
\section{Loading and processing the
data}\label{loading-and-processing-the-data}}

\hypertarget{loading-the-data}{%
\subsection{Loading the data}\label{loading-the-data}}

\begin{Shaded}
\begin{Highlighting}[]
\KeywordTok{library}\NormalTok{(readxl)}
\NormalTok{Data <-}\StringTok{ }\KeywordTok{read.csv}\NormalTok{(}\StringTok{"activity.csv"}\NormalTok{)}
\KeywordTok{str}\NormalTok{(Data)}
\end{Highlighting}
\end{Shaded}

\begin{verbatim}
## 'data.frame':    17568 obs. of  3 variables:
##  $ steps   : int  NA NA NA NA NA NA NA NA NA NA ...
##  $ date    : chr  "2012-10-01" "2012-10-01" "2012-10-01" "2012-10-01" ...
##  $ interval: int  0 5 10 15 20 25 30 35 40 45 ...
\end{verbatim}

\hypertarget{what-is-mean-total-number-of-steps-taken-per-day}{%
\section{What is mean total number of steps taken per
day?}\label{what-is-mean-total-number-of-steps-taken-per-day}}

\begin{Shaded}
\begin{Highlighting}[]
\KeywordTok{library}\NormalTok{(dplyr)}
\end{Highlighting}
\end{Shaded}

\begin{verbatim}
## 
## Attaching package: 'dplyr'
\end{verbatim}

\begin{verbatim}
## The following objects are masked from 'package:stats':
## 
##     filter, lag
\end{verbatim}

\begin{verbatim}
## The following objects are masked from 'package:base':
## 
##     intersect, setdiff, setequal, union
\end{verbatim}

\begin{Shaded}
\begin{Highlighting}[]
\NormalTok{total.steps <-}\StringTok{ }\KeywordTok{tapply}\NormalTok{(Data}\OperatorTok{$}\NormalTok{steps, Data}\OperatorTok{$}\NormalTok{date, }\DataTypeTok{FUN=}\NormalTok{sum, }\DataTypeTok{na.rm=}\OtherTok{TRUE}\NormalTok{)}
\KeywordTok{head}\NormalTok{(total.steps, }\DataTypeTok{n =} \DecValTok{5}\NormalTok{)}
\end{Highlighting}
\end{Shaded}

\begin{verbatim}
## 2012-10-01 2012-10-02 2012-10-03 2012-10-04 2012-10-05 
##          0        126      11352      12116      13294
\end{verbatim}

\hypertarget{histogram}{%
\subsection{1. Histogram}\label{histogram}}

\begin{Shaded}
\begin{Highlighting}[]
\KeywordTok{hist}\NormalTok{(total.steps,}
     \DataTypeTok{col =} \StringTok{"Purple"}\NormalTok{,}
     \DataTypeTok{main =} \StringTok{"Histogram"}\NormalTok{,}
     \DataTypeTok{xlab =} \StringTok{"Total number of steps"}\NormalTok{,}
     \DataTypeTok{ylab =} \StringTok{"Frequency"}\NormalTok{,}
     \DataTypeTok{breaks =} \DecValTok{30}\NormalTok{)}
\end{Highlighting}
\end{Shaded}

\includegraphics{PA1_template_files/figure-latex/unnamed-chunk-2-1.pdf}

\hypertarget{mean-and-media}{%
\subsection{2. Mean and media}\label{mean-and-media}}

\begin{Shaded}
\begin{Highlighting}[]
\NormalTok{steps_Mean <-}\StringTok{ }\KeywordTok{mean}\NormalTok{(total.steps, }\DataTypeTok{na.rm =} \OtherTok{TRUE}\NormalTok{)}
\NormalTok{steps_Median <-}\StringTok{ }\KeywordTok{median}\NormalTok{(total.steps, }\DataTypeTok{na.rm =} \OtherTok{TRUE}\NormalTok{)}
\end{Highlighting}
\end{Shaded}

\begin{itemize}
\tightlist
\item
  Mean: \texttt{steps\_Mean}
\item
  Media: \texttt{steps\_Media}
\end{itemize}

\hypertarget{what-is-the-average-daily-activity-pattern}{%
\section{What is the average daily activity
pattern?}\label{what-is-the-average-daily-activity-pattern}}

\begin{Shaded}
\begin{Highlighting}[]
\NormalTok{averages <-}\StringTok{ }\KeywordTok{aggregate}\NormalTok{(}\DataTypeTok{x =} \KeywordTok{list}\NormalTok{(}\DataTypeTok{steps =}\NormalTok{  Data}\OperatorTok{$}\NormalTok{steps), }
                      \DataTypeTok{by =} \KeywordTok{list}\NormalTok{(}\DataTypeTok{interval =}\NormalTok{ Data}\OperatorTok{$}\NormalTok{interval), }
                      \DataTypeTok{FUN =}\NormalTok{ mean, }
                      \DataTypeTok{na.rm =} \OtherTok{TRUE}\NormalTok{)}
\KeywordTok{head}\NormalTok{(averages, }\DataTypeTok{n =} \DecValTok{5}\NormalTok{)}
\end{Highlighting}
\end{Shaded}

\begin{verbatim}
##   interval     steps
## 1        0 1.7169811
## 2        5 0.3396226
## 3       10 0.1320755
## 4       15 0.1509434
## 5       20 0.0754717
\end{verbatim}

\hypertarget{time-series-plote}{%
\subsection{1. Time series plote}\label{time-series-plote}}

\begin{Shaded}
\begin{Highlighting}[]
\KeywordTok{library}\NormalTok{(ggplot2)}
\KeywordTok{ggplot}\NormalTok{(}\DataTypeTok{data =}\NormalTok{ averages, }\KeywordTok{aes}\NormalTok{(}\DataTypeTok{x =}\NormalTok{ interval, }\DataTypeTok{y =}\NormalTok{ steps)) }\OperatorTok{+}
\StringTok{  }\KeywordTok{geom_line}\NormalTok{(}\DataTypeTok{color =} \StringTok{"purple"}\NormalTok{, }\DataTypeTok{size =} \DecValTok{2}\NormalTok{) }\OperatorTok{+}
\StringTok{  }\KeywordTok{labs}\NormalTok{(}\DataTypeTok{title =} \StringTok{"Time series plot"}\NormalTok{) }\OperatorTok{+}
\StringTok{  }\KeywordTok{theme}\NormalTok{(}\DataTypeTok{plot.title =} \KeywordTok{element_text}\NormalTok{(}\DataTypeTok{hjust =} \FloatTok{0.5}\NormalTok{, }\DataTypeTok{face =} \StringTok{"bold"}\NormalTok{, }\DataTypeTok{size =} \DecValTok{12}\NormalTok{)) }\OperatorTok{+}
\StringTok{  }\KeywordTok{xlab}\NormalTok{(}\StringTok{"5 minutes interval"}\NormalTok{) }\OperatorTok{+}
\StringTok{  }\KeywordTok{ylab}\NormalTok{(}\StringTok{"Number of steps"}\NormalTok{)}
\end{Highlighting}
\end{Shaded}

\includegraphics{PA1_template_files/figure-latex/unnamed-chunk-5-1.pdf}

\hypertarget{interval-with-the-maximun-number-of-steps}{%
\subsection{2. Interval with the maximun number of
steps}\label{interval-with-the-maximun-number-of-steps}}

\begin{Shaded}
\begin{Highlighting}[]
\NormalTok{averages[}\KeywordTok{which.max}\NormalTok{(averages}\OperatorTok{$}\NormalTok{steps),]}
\end{Highlighting}
\end{Shaded}

\begin{verbatim}
##     interval    steps
## 104      835 206.1698
\end{verbatim}

\hypertarget{imputing-missing-values}{%
\section{Imputing missing values}\label{imputing-missing-values}}

here are a number of days/intervals where there are missing values
(coded as \color{red}{\verb|NA|}NA). The presence of missing days may
introduce bias into some calculations or summaries of the data

\hypertarget{total-number-of-missing-values}{%
\subsection{1. Total number of missing
values}\label{total-number-of-missing-values}}

\begin{Shaded}
\begin{Highlighting}[]
\NormalTok{missing_values <-}\StringTok{ }\KeywordTok{length}\NormalTok{(}\KeywordTok{which}\NormalTok{(}\KeywordTok{is.na}\NormalTok{(Data}\OperatorTok{$}\NormalTok{steps)))}
\NormalTok{missing_values}
\end{Highlighting}
\end{Shaded}

\begin{verbatim}
## [1] 2304
\end{verbatim}

\hypertarget{strategy-for-filling-values-in-the-data-set}{%
\subsection{2. Strategy for filling values in the data
set}\label{strategy-for-filling-values-in-the-data-set}}

In order to fill the data set with the missing data we are going to do
the following steps:

\begin{itemize}
\tightlist
\item
  Create the variable ``interval\_steps'' in which we fill the missing
  values with the mean with the R function ``apply - tapply''
\item
  Create the variable ``Data split'' which is the split activity data by
  intervals
\item
  Fill the missing data for each interval
\end{itemize}

\hypertarget{new-dataset}{%
\subsection{3. New dataset}\label{new-dataset}}

\begin{Shaded}
\begin{Highlighting}[]
\NormalTok{Interval_steps <-}\StringTok{ }\KeywordTok{tapply}\NormalTok{(Data}\OperatorTok{$}\NormalTok{steps, Data}\OperatorTok{$}\NormalTok{interval, mean, }\DataTypeTok{na.rm =} \OtherTok{TRUE}\NormalTok{)}
\NormalTok{Data_split <-}\StringTok{ }\KeywordTok{split}\NormalTok{(Data, Data}\OperatorTok{$}\NormalTok{interval)}
\ControlFlowTok{for}\NormalTok{(i }\ControlFlowTok{in} \DecValTok{1}\OperatorTok{:}\KeywordTok{length}\NormalTok{(Data_split))\{}
\NormalTok{    Data_split[[i]]}\OperatorTok{$}\NormalTok{steps[}\KeywordTok{is.na}\NormalTok{(Data_split[[i]]}\OperatorTok{$}\NormalTok{steps)] <-}\StringTok{ }\NormalTok{Interval_steps[i]}
\NormalTok{\}}
\NormalTok{activity.imputed <-}\StringTok{ }\KeywordTok{do.call}\NormalTok{(}\StringTok{"rbind"}\NormalTok{, Data_split)}
\NormalTok{activity.imputed <-}\StringTok{ }\NormalTok{activity.imputed[}\KeywordTok{order}\NormalTok{(activity.imputed}\OperatorTok{$}\NormalTok{date) ,]}
\KeywordTok{head}\NormalTok{(activity.imputed, }\DataTypeTok{n =} \DecValTok{5}\NormalTok{)}
\end{Highlighting}
\end{Shaded}

\begin{verbatim}
##          steps       date interval
## 0.1  1.7169811 2012-10-01        0
## 5.2  0.3396226 2012-10-01        5
## 10.3 0.1320755 2012-10-01       10
## 15.4 0.1509434 2012-10-01       15
## 20.5 0.0754717 2012-10-01       20
\end{verbatim}

\hypertarget{histogram-of-the-total-number-of-days}{%
\subsection{4. Histogram of the total number of
days}\label{histogram-of-the-total-number-of-days}}

To be able to visualize the total number of steps taken each day first
it i necessary to calculate the total steps with impute values.

\begin{Shaded}
\begin{Highlighting}[]
\NormalTok{StepsPerDay.imputed <-}\StringTok{ }\KeywordTok{tapply}\NormalTok{(activity.imputed}\OperatorTok{$}\NormalTok{steps, activity.imputed}\OperatorTok{$}\NormalTok{date, sum)}
\end{Highlighting}
\end{Shaded}

With the data the next step is to plot the results

\begin{Shaded}
\begin{Highlighting}[]
\KeywordTok{hist}\NormalTok{(StepsPerDay.imputed,}
     \DataTypeTok{col =} \StringTok{"Purple"}\NormalTok{,}
     \DataTypeTok{main =} \StringTok{"Histogram: Steps per Day"}\NormalTok{,}
     \DataTypeTok{xlab =} \StringTok{"Number of Steps"}\NormalTok{,}
     \DataTypeTok{breaks =} \DecValTok{15}\NormalTok{)}
\end{Highlighting}
\end{Shaded}

\includegraphics{PA1_template_files/figure-latex/unnamed-chunk-10-1.pdf}
With the new data, we want to see if the total daily number of spetps
changes imputing missing values. To be able to perform that, we need to
calculate the mean and media of the new data set

\begin{Shaded}
\begin{Highlighting}[]
\NormalTok{Mean_imputed <-}\StringTok{ }\KeywordTok{mean}\NormalTok{(StepsPerDay.imputed, }\DataTypeTok{na.rm =} \OtherTok{TRUE}\NormalTok{)}
\NormalTok{Median_imputed <-}\StringTok{ }\KeywordTok{median}\NormalTok{(StepsPerDay.imputed, }\DataTypeTok{na.rm =} \OtherTok{TRUE}\NormalTok{)}
\end{Highlighting}
\end{Shaded}

\begin{itemize}
\tightlist
\item
  Mean: \texttt{Mean\_imputed}
\item
  Media: \texttt{Median\_imputed}
\end{itemize}

\hypertarget{are-there-differences-in-activity-patterns-between-weekdays-and-weekends}{%
\section{Are there differences in activity patterns between weekdays and
weekends?}\label{are-there-differences-in-activity-patterns-between-weekdays-and-weekends}}

\hypertarget{new-factor-variable-in-the-dataset-with-two-levels-weekday-and-weekend}{%
\subsection{1. New factor variable in the dataset with two levels --
``weekday'' and
``weekend''}\label{new-factor-variable-in-the-dataset-with-two-levels-weekday-and-weekend}}

\begin{Shaded}
\begin{Highlighting}[]
\KeywordTok{library}\NormalTok{(dplyr)}
\NormalTok{Data_new <-}\StringTok{ }\KeywordTok{mutate}\NormalTok{(activity.imputed, }\DataTypeTok{weektype =} \KeywordTok{ifelse}\NormalTok{(}\KeywordTok{weekdays}\NormalTok{(}\KeywordTok{as.Date}\NormalTok{(activity.imputed}\OperatorTok{$}\NormalTok{date)) }\OperatorTok{==}\StringTok{ "Saturday"} \OperatorTok{|}\StringTok{ }\KeywordTok{weekdays}\NormalTok{(}\KeywordTok{as.Date}\NormalTok{(activity.imputed}\OperatorTok{$}\NormalTok{date)) }\OperatorTok{==}\StringTok{ "Sunday"}\NormalTok{, }\StringTok{"weekend"}\NormalTok{, }\StringTok{"weekday"}\NormalTok{))}
\NormalTok{Data_new}\OperatorTok{$}\NormalTok{weektype <-}\StringTok{ }\KeywordTok{as.factor}\NormalTok{(Data_new}\OperatorTok{$}\NormalTok{weektype)}
\KeywordTok{head}\NormalTok{(Data_new)}
\end{Highlighting}
\end{Shaded}

\begin{verbatim}
##       steps       date interval weektype
## 1 1.7169811 2012-10-01        0  weekday
## 2 0.3396226 2012-10-01        5  weekday
## 3 0.1320755 2012-10-01       10  weekday
## 4 0.1509434 2012-10-01       15  weekday
## 5 0.0754717 2012-10-01       20  weekday
## 6 2.0943396 2012-10-01       25  weekday
\end{verbatim}

\hypertarget{section}{%
\subsection{2.}\label{section}}

\begin{Shaded}
\begin{Highlighting}[]
\NormalTok{interval_full <-}\StringTok{ }\NormalTok{Data_new }\OperatorTok
\StringTok{  }\KeywordTok{group_by}\NormalTok{(interval, weektype) }\OperatorTok
\StringTok{  }\KeywordTok{summarise}\NormalTok{(}\DataTypeTok{steps =} \KeywordTok{mean}\NormalTok{(steps))}
\end{Highlighting}
\end{Shaded}

\begin{verbatim}
## `summarise()` regrouping output by 'interval' (override with `.groups` argument)
\end{verbatim}

\begin{Shaded}
\begin{Highlighting}[]
\KeywordTok{ggplot}\NormalTok{(interval_full, }\KeywordTok{aes}\NormalTok{(}\DataTypeTok{x=}\NormalTok{interval, }\DataTypeTok{y=}\NormalTok{steps, }\DataTypeTok{color =}\NormalTok{ weektype)) }\OperatorTok{+}
\StringTok{  }\KeywordTok{geom_line}\NormalTok{(}\DataTypeTok{size =} \DecValTok{2}\NormalTok{) }\OperatorTok{+}
\StringTok{  }\KeywordTok{facet_wrap}\NormalTok{(}\OperatorTok{~}\NormalTok{weektype, }\DataTypeTok{ncol =} \DecValTok{1}\NormalTok{, }\DataTypeTok{nrow=}\DecValTok{2}\NormalTok{)}
\end{Highlighting}
\end{Shaded}

\includegraphics{PA1_template_files/figure-latex/unnamed-chunk-13-1.pdf}

\end{document}
